\documentclass[a4paper,12pt]{article}

\usepackage[utf8]{inputenc}
\usepackage[T1]{fontenc}

\usepackage[notes=true]{lib/dtrt}
\usepackage{algorithm}
\usepackage{algpseudocode}
\usepackage{amsmath}
\usepackage{xcolor}
\usepackage{enumitem}
\usepackage{graphicx}
\usepackage{todonotes}
\usepackage{listings}
\usepackage{pxfonts}
\usepackage[margin=1in]{geometry}

\usepackage{lib/lang}  % include language definition for protobuf
\usepackage{lib/style} % include custom style for proto declarations.	

%\usepackage{sagetex}

\makeatletter
\newcommand{\subparhead}[1]{\smallskip \noindent {\itshape\ignorespaces \dt@MaybeAddPunct{#1}}\hskip 0.9em plus 0.3em minus 0.3em \dt@ignorespacesandimplicitepars}
\makeatother

% Auuthor notes (using dtrt's macros). Switch the dtrt package flag to notes=false to hide.
\definecolor{darkgreen}{rgb}{0,0.6,0}
\newcommand{\dnote}[1]{\dtcolornote[Daniel]{red}{#1}}
\newcommand{\anote}[1]{\dtcolornote[Aurell]{blue}{#1}}
\newcommand{\enote}[1]{\dtcolornote[Eran]{darkgreen}{#1}}

\newcommand\dtodo[1]{\todo[color=red!20]{#1}}
\newcommand\atodo[1]{\todo[color=green!20]{#1}}
\newcommand\etodo[1]{\todo[color=blue!20]{#1}}
\def\boxit#1{%
	\smash{\color{red}\fboxrule=1pt\relax\fboxsep=-2pt\llap{\rlap{\fbox{\strut\makebox[#1]{}}}~}}\ignorespaces
}

% Style for identifiers.
\newcommand{\textid}[1]{\texttt{#1}}
% Note paragraphs.
\newcommand{\noteparagraph}[1]{\textbf{#1}.}

%\makeatletter
%\def\BState{\State\hskip-\ALG@thistlm}
%\makeatother

\newenvironment{definition}{\parhead{Definition}}{} % XXX

%%%%%%%%%%%%%%%%%%%%%%%%%%%%%%%%%%%%%%%%%%%%%%%%%%%%%%%%%%%%%%%%%%%%%%%%%%%%%%
% NP statements
\definecolor{NPcolor}{rgb}{0,0.3,0}

\newcommand{\NPstatement}[3]{
{\color{NPcolor}
\begin{itemize}[nosep]
\item[] #1;
\item[] #2;
\item[] #3.
\end{itemize}
}
}

\newcommand{\NPstatementNOperiod}[3]{
{\color{NPcolor}
\begin{itemize}[nosep]
\item[] #1;
\item[] #2;
\item[] #3
\end{itemize}
}
}

%%%%%%%%%%%%%%%%%%%%%%%%%%%%%%%%%%%%%%%%%%%%%%%%%%%%%%%%%%%%%%%%%%%%%%%%%%%%%%
% Constraint systems

\newcommand{\defem}[1]{\emph{\textbf{#1}}}
\newcommand{\enlargeparens}{{\rule{0pt}{2.2ex}}}
\newcommand{\ip}[2]{\left\langle #1 , #2 \right\rangle}
\newcommand{\bigip}[2]{\ip{\enlargeparens #1}{#2}}

\newcommand{\FieldSize}{q}
\newcommand{\FieldBits}{n}
\newcommand{\Field}{\mathbb{F}_{\FieldSize}} %_\FieldSize
\newcommand{\NumConstr}{n_{\mathsf{c}}}
\newcommand{\NumWit}{{n_{\mathsf{w}}}}
\newcommand{\NumInst}{{n_{\mathsf{x}}}}
\newcommand{\NumConc}{{n_{\mathsf{z}}}}
\newcommand{\inst}{x}
\newcommand{\wit}{w}
\newcommand{\conc}{z}
\newcommand{\vinst}{\vec{\inst}}
\newcommand{\vwit}{\vec{\wit}}
\newcommand{\vconc}{\vec{z}}
\newcommand{\lcoef}{a}
\newcommand{\rcoef}{b}
\newcommand{\ocoef}{c}
\newcommand{\System}{\mathcal{S}}
\newcommand{\Rel}{\mathcal{R}}
\newcommand{\Lang}{\mathcal{L}}

\newcommand{\zo}{\{0,1\}}
\newcommand{\NN}{\mathbb{N}}
\newcommand{\ZZ}{\mathbb{Z}}

%%%%%%%%%%%%%%%%%%%%%%%%%%%%%%%%%%%%%%%%%%%%%%%%%%%%%%%%%%%%%%%%%%%%%%%%%%%%%%

%opening
\title{zkInterface, a standard tool for zero-knowledge interoperability}
\author{Daniel Benarroch, Kobi Gurkan, Aurélien Nicolas, Eran Tromer}
	 
\begin{document}

\maketitle

\enote{Add abstract}
\begin{abstract}
Zero Knowledge Proof (ZKP) systems are usually represented in 
\end{abstract}

\tableofcontents
%=============================================================================
\section{Overview}
This standard, part of the ZKProof Standardization effort \cite{ZKProofSecurity, ZKProofImplementation, ZKProofApplications}, aims to facilitate interoperability between zero knowledge proof implementations, at the level of the low-constraint systems that are produced by frontends (and represent application-level statements) and consumed by cryptographic backends (which generate and verify the proofs). The high-level goal is to enable decoupling of frontends from backends, allowing application writers to choose the frontend most convenient for their functional and development needs and combine it with the backend that best matches their performance and security needs. This includes communicating constraint systems, communicating variable assignments (for production of proofs), and also construction of constraint systems out of smaller building blocks (gadgets) possibly implemented by different authors and frameworks.

This first revision focuses on non-interactive proof systems (NIZKs) for general statements (i.e., NP relations) represented in the R1CS/QAP-style constraint system representation\footnote{See Appendix \ref{sec:r1cs-def} for a definition of Rank 1 Constraint System}. This includes many, though not all, of the practical general-purpose ZKP schemes currently deployed. While this focus allows us to define concrete formats for interoperability, we recognize that additional constraint system representation styles (e.g., arithmetic and Boolean circuits and algebraic constraints) are in use, and are within scope of future revisions.

An implementation of the zkInterface can be found in the following GitHub repository \href{https://github.com/QED-it/gadget_standard}{QED-it/gadget\_standard}.
%-----------------------------------------------------------------------------
\subsection{Background}

Zero-Knowledge Proofs are cryptographic primitives that allow some entity (the prover) to prove to another party (the verifier) the validity of some statement or relation. Today there are many efficient constructions of NIZKs, each with different trade-offs, as well as several implementations of the proving systems. By standardizing zero-knowledge proofs, we aim to foster the proper use of the technology.

Every proving system can be divided \cite{ZKProofImplementation} into the backend, which is the portion of the software that contains the implementation of the underlying cryptographic protocol, and the frontend, which provides means to express statements in a convenient language, allowing to prove such statements in zero knowledge by compiling them into a low-level representation of the statement.

The backend of a proving system consists of the key generation, proving and verification algorithms. It proves statements where the instance and witness are expressed as variable assignments, and relations are expressed via low-level languages (such as arithmetic circuits, Boolean circuits, R1CS/QAP constraint systems or arithmetic constraint satisfaction problems). There are numerous such backends, including implementations of many of the schemes discussed in the Security Track proceeding \cite{ZKProofSecurity}.

The frontend consists of the following:
\begin{itemize}
	\item The specification of a high-level language for expressing statements.
	\item A compiler that converts relations expressed in the high-level language into the low-level relations suitable for some backend(s). For example, this may produce an R1CS constraint system.
	\item Instance reduction: conversion of the instance in a high-level statement to a low-level instance (e.g., assignment to R1CS instance variables).
	\item Witness reduction: conversion of the witness to a high-level statement to a low-level witness (e.g., assignment to witness variables).
	\item Typically, a library of "gadgets" consisting of useful and hand-optimized building blocks for statements.
\end{itemize}

Since the offerings and features of backends and frontends evolve rapidly, we refer the reader to the curated taxonomy at \url{https://zkp.science} for the latest information.  

Currently, existing frontend are implemented to work best with their corresponding backend, the proving system is usually built end-to-end. The frontend compiles a statement into the native representation used by the cryptographic protocol in the backend, in many cases without explicitly exposing the constraint system compilation to the user. Moreover, if the compilers can output intermediary files and configurations, they are usually in a non-standard format. In practice this means that
\begin{itemize}
	\item There is no portability between different backends and frontends, and
	\item It is not possible to generate a constraint system using different frontends
\end{itemize}   

With this proposal we aim to solve this by creating an interface between frontends and backends, as seen in Figure \ref{interface}. We add an explicit formatting layer between the frontends and backends that allows the user to ``pick-and-chose'' which existing frontend and backend they prefer. Furthermore, given the programmatic design of our interface, a specific gadget, can itself call a sub-gadget from a different frontend. This enables the use of more than one frontend to generate the complete statement. At present this format is tailored for the R1CS statement representation, defined in Appeneix~\ref{sec:r1cs-def} and discussed below.

\begin{figure}[h!]
	\includegraphics[width=\linewidth]{graphics/interop.png}
	\caption{Interoperability between frontends and backends with zkInterface}
	\label{interface}
\end{figure}

%-----------------------------------------------------------------------------
\subsection{Terminology}
The terminology follows the Implementation Track proceeding \cite{ZKProofImplementation}, and any new terms and concepts will be defined accordingly. The R1CS statement representation is introduced and defined in Appeneix~\ref{sec:r1cs-def}.

\subsection{Goals} 
\label{goals}

There are several forms of interoperability and we set some of these as goals for this standard. One such form is between different implementations of the same backend construction, providing an interoperable format for the proving and verification keys, as well as the proof. Another form is between backends and frontends, which is the focus of this standard. The following are stronger forms of the latter kind of interoperability which have been identified as desirable by practitioners.

\parhead{Statement instance and witness formats} 
Specifying a standard format for the statement instance and witness enables users to have their choice of frameworks (frontends and backends) and streaming for storage and communication, and facilitate creation of benchmark test cases that could be executed by any backend accepting these formats.
 
Crucially, analogous formats are desired for constraint system languages other than R1CS.

\parhead{Statement semantics, variable representation and mapping}

Beyond the above, there’s a need for different implementations to coordinate the semantics of the statement (instance) representation of constraint systems. For example, a high-level protocol may have an RSA signature as part of the statement, leaving ambiguity on how big integers modulo a constant are represented as a sequence of variables over a smaller field, and at what indices these variables are placed in the actual R1CS instance.

Precise specification of statement semantics, in terms of higher-level abstraction, is needed for interoperability of constraint systems that are invoked by several different implementations of the instance reduction (from high-level statement to the actual input required by the ZKP prover and verifier). One may go further and try to reuse the actual implementation of the instance reduction, taking a high-level and possibly domain-specific representation of values (e.g., big  integers) and converting it into low-level variables. This raises questions of language and platform incompatibility, as well as proper modularization and packaging.

Note that correct statement semantics is crucial for security. Two implementations that use the same high-level protocol, same constraint system and compatible backends may still fail to correctly interoperate if their instance reductions are incompatible -- both in completeness (proofs don’t verify) or soundness (causing false but convincing proofs, implying a security vulnerability). Moreover, semantics are a requisite for verification and helpful for debugging. Beyond interoperability, some low-level building blocks (e.g., finite field and elliptic curve arithmetic) are needed by many or all implementations, and suitable libraries can be reused.

Some backends can exploit uniformity or regularity in the constraint system (e.g., repeating patterns or algebraic structure), and could thus take advantage of formats and semantics that convey the requisite information.

Given the typical complexity level of today’s constraint systems, it is often acceptable to handle all of the above manually, by fresh re-implementation based on informal specifications and inspection of prior implementation. Our goal, however, is for the interface to handle the semantics of the gadgets, reducing the predisposition to error as application complexity grows. The following paragraphs expand on how the semantics should be considered for interoperability of gadgets.

\parhead{Witness reduction}
Similar considerations arise for the witness reduction, mapping a high-level witness representation for a given statement into the assignment to witness variables (as defined by the instance). For example, a high-level protocol may use Merkle trees of particular depth with a particular hash function, and a high-level instance may include a Merkle authentication path. The witness reduction would need to convert these into witness variables, that contain all of the Merkle authentication path data encoded by some particular convention into field elements and assigned in some particular order. Moreover, it would also need to convert the numerous additional witness variables that occur in the constraints that evaluate the hash function, ensure consistency and Booleanity, amonh others.

\parhead{Gadgets interoperability}
Beyond using fixed, monolithic constraint systems and their assignments, there is a need for sharing subcircuits and gadgets. For example, libsnark offers a rich library of R1CS gadgets, which developers of several front-end compilers would like to reuse in the context of their own constraint-system construction framework. 

While porting chunks of constraints across frameworks is relatively straightforward, there are challenges in coordinating the semantics of the externally-visible variables of the gadget, analogous to but more difficult than those mentioned above for full constraint systems. Mainly, there is a need to coordinate or reuse the semantics of a gadget’s externally-visible variables (those accessible by other gadgets), as well as the witness reduction function of imported gadgets in order to assign a witness into the internal variables of the gadget.

As for instance semantics, well-defined gadget semantics is crucial for soundness, completeness and verification, and is helpful for debugging.

\parhead{Procedural interoperability}
An attractive approach to the aforementioned needs for instance and witness reductions (both at the level of whole constraint systems and at the gadget level) is to enable one implementation to invoke the instance/witness reductions of another, even across frameworks and programming languages.

This requires communication not of mere data, but invocation of procedural code. Suggested approaches to this include linking against executable code (e.g., .so files or .dll), using some elegant and portable high-level language with its associated portable, or using a low-level portable executable format such as WebAssembly. All of these require suitable calling conventions (e.g., how are field elements represented?), usage guidelines and examples.


\subsection{Desiderata}
The following requirements that guide our design, within the large space of possibilities for achieving the above goals.

\begin{enumerate}
	\item Interoperability across frontend frameworks and programming languages.
	\item Ability to write gadgets that can be consumed by different frontends and backends.
	\item Minimize copying and duplication of data.
	\item The overhead of the R1CS construction and witness reduction should be low (and in particular, linear) compared to a native implementation of the same gadgets in existing frameworks.
	\item Expose details of the backend's interface that are necessary for performance (e.g., constraint system representation and algebraic fields).
	\item The approach can be extended to support constraint systems beyond R1CS.
\end{enumerate}

\subsection{Scope, limitations and possible extensions}

We present a set of specifications to be standardized to enable the use of an interface between zero-knowledge proof systems. We have identified the minimal items needed to create a standard interface that meets the goals and desired requirements. The following points form the scope of our proposed standard.

\begin{itemize}
	\item Standard defined messages that the caller and callee exchange.
	\item The serialization of the messages.
	\item A protocol to build a constraint system from gadget composition.
	\item Technical recommendations for implementation.
\end{itemize}

Some limitations of the standard, with respect to interoperability, are the following.

\parhead{Limitations.}
The following are not addressed by this standard:

\subparhead{Backend interoperability} We do not aim to standardize the proof algorithms, the format of the proofs generated by a backend, or the format of the proving and verification keys -- all of which would be required to achieve interoperability between backends. (See ``Proof interoperability'' and ``Common reference strings'' in \cite{ZKProofImplementation}.

\subparhead{Programming language and frontend frameworks} We are intentionally agnostic about, and do not aim to standardize, the programming language and programming framework used by frontends.

\parhead{Extensions.}
The following are not covered by the present revision of the standard, but should be covered by future extensions. Thus, the standard should be flexible enough and easy to extend in a backward-compatible compatible fashion, to achieve the following:

\subparhead{Other constraint system representations.}
Going beyond R1CS, we plan on supporting other constraint system representations that are native to some ZK backends. These include Boolean circuits, arithmetic circuits, and algebraic constraint systems.

\subparhead{Uniform constraint systems}
Some backends can take advantage of uniformity in the constraint system (e.g., when some elements are repeated many times). Support the expression of such uniformity in the constraint system representation, so backends can utilize it.

\subparhead{Packaging} Describe self-contained packaging of a gadget would allow for portable execution of the gadgets on different platforms.

\subparhead{Typing} Enforcing properties of variables using a type system (e.g., a ``boolean variable'' type that ensures a variable is 0 or 1 even when working over a large field).

%=============================================================================
\section{Design}

%-----------------------------------------------------------------------------
\subsection{Approach}

\enote{Explain data flow; calling approach based on passing buffers; in-process or cross-process. Explain using FlatBuffers and why. Explain in general terms that we deal directly with variables in a global space, with lightweight coordination to allocate variables; and why (avoid duplication/rewriting etc.}

zkInterface is a procedural, purely functional interface for zero-knowledge systems that enables cross-language interoperability via dynamic linking and shared memory. The current version, even if limiting, creates an interface based on R1CS formatting and offers the ability to abstractly craft a constraint system building from different components, possibly written in different frameworks, by determining how data should be written and read. 

\enote{I'm not sure what this means:}
It can also be seen as a design tool for improved generation of constraints and usability, analogous to a portable binary format, since one can parametrize the functions calls and easily compose different functions, or components, that are not directly compatible.

\begin{figure}[h!]
	\includegraphics[width=\linewidth]{call_flow.png}
	\caption{The flow of messages between libraries using the interface}
	\label{flow}
\end{figure}


\subsection{Architecture}

\enote{Describe the high-level design, introducing all the main concepts in a readable narrative and referring to them concretely by the  identifier name in the sourcecode.}

\paragraph{Interface.}

	The interface is defined as a set of messages that the caller and callee can exchange.
	The definition is provided in Listing \ref{gadget.fbs} as a FlatBuffers schema.
	Refer to the inline documentation of each message and field.

	\subparagraph{Note}
	The FlatBuffers system includes a simple interface definition language,
	a data layout specification,
	a clear evolution path for future extensions of the standard,
	support for all common programming languages,
	and the possibility of very efficient implementations.
	The specification of FlatBuffers can be found at
	\href{https://google.github.io/flatbuffers/}{https://google.github.io/flatbuffers/}.

\paragraph{Messages Flow.}

	The flow of messages is illustrated in Figure \ref{flow}.

	The caller calls the component code with a single Call message.
	The component exits with a single Return message.
	This is a control flow analoguous to a function call in common programming languages.

	The caller also provides a way for the component to send 
	R1CSConstraints and AssignedVariables messages.
	This is an output channel distinct from the return message.

	During instance reduction,
	a component may add any number of constraints to the constraint system
	by sending one or more R1CSConstraints messages.
	The caller and other components may do so as well.

	During witness reduction,
	a component may assign values to variables
	by sending one or more AssignedVariables messages.

	\subparagraph{Note}
	The design of constraints and assignments channels
	allows a component to call subcomponents itself.
	Messages from all (sub-)components can simply be sent separately
	without the need to aggregate them into a single message.
	
	Moreover, an implementation can decouple the proving system
	from the logic of building constraints and assignments,
	by arranging for the constraints and assignments messages
	to be processed by the proving system, independently from the control logic.

\paragraph{Variables.}

	All variables in a constraint system are assigned a numerical identifier
	unique within this system.
	Messages that contain constraints or assignments refer to variables by their
	unique ID.

	\subparagraph{Note}
	This design allows implementations to aggregate and handle messages in a generic way,
	without any	reference to the components or mechanisms that generated them.

	\anote{Must be consecutive or are gaps allowed?}

\paragraph{Local Variables Allocation.}

	A component may allocate a number of local variables to use
	in the internal implementation of the function that it computes.
	They are analoguous to stack variables in common programming languages.

	The following protocol is used to allocate variable IDs that are
	unique within a whole constraint system.
	\begin{itemize}
		\item The caller must provide a numerical ID greater than all IDs that have already been allocated, called the Free-Before ID.
		\item The component may use the Free-Before ID and consecutive IDs as its local variables IDs.
		\item The component must return the next consecutive ID that it did not use, called the Free-After ID.
		\item The caller must treat IDs lesser than the Free-After ID as allocated by the component,
			and must not use them.
	\end{itemize}

	During instance reduction, the component can refer to
	its local variables in the R1CSConstraints messages that it generates.
	The caller and other parts of the program must not refer to these local variables.

	During witness reduction, the component must assign values to its local variables
	by sending AssignedVariables messages.

\paragraph{Incoming/Outgoing Variables.}

	The concept of incoming, outgoing variables arises when a program is decomposed into components.
	These variables serve as the functional interface between a component and its caller.
	They are analoguous to arguments and return values of functions in common programming languages.
	A variable is not inherently incoming, outgoing, nor local;
	rather, this is a convention in the context of a component call.

	The caller provides the IDs of variables to be used as incoming and outgoing variables by the component.
	There may be no outgoing variables if the component implements a pure assertion.

	During instance reduction, both the caller and the component can refer to
	these variables in the R1CSConstraints messages that they generate.
	Other parts of the program may also refer to these same variables in their own contexts.

	During witness reduction, the caller must pass incoming values to the component in the Call message.
	The component must return outgoing values to the caller in the Return message.

	The caller is responsible for the assignment of values to both incoming and outgoing variables.
	How this is achieved depends on the caller and proving system,
	and on whether some variables are treated as public inputs of the zero-knowledge program.

\paragraph{C Interface.}

	A C interface to exchange messages is defined in Listing \ref{gadget.h}.
	Refer to the inline documentation.

\paragraph{Stream and File Format.}
	\enote{Explain what's already induced by FlatBuffers: explain that the framework, plus our .fbs file, specify how the aforementioned high-level messages are actually represented at the byte level, including message framing, message naming, etc..}
	All messages are framed, meaning that they can be concatenated and distinguished in streams of bytes or in files.
	Messages must be prefixed by the size of the message not including the prefix,
	as a 4-bytes little-endian unsigned integer.
	\enote{Say something about streams - stdio/stdout, files, TCP connections, etc.?}



\subsection{Interface Definition}

A copy of the messages definition is hosted at \\
\href{https://github.com/QED-it/gadget_standard/blob/master/zkinterface.fbs}{https://github.com/QED-it/gadget\_standard/blob/master/zkinterface.fbs}

\lstinputlisting[
	caption=Messages Definition,
	language=flatbuffers2,style=protobuf,
	label=zkinterface.fbs]{../zkinterface.fbs}


%=============================================================================
\section{Implementation}
\label{sec:implementation}

The above section describes a protocol and the format of messages.
Applications can execute, and exchange messages with
gadgets and proving systems implementations in a variety of ways.
We recommend two approaches in this section.

\parhead{In-memory execution}
\label{inmemory}
The application may execute the code of a component in its own process.

The component exposes its functionnality as a function calleable using the C calling convention of the platform. The component code may be linked statically or be loaded from a shared library.

The application must prepare a \textid{ComponentCall} message in memory, and implement one callback functions to receive the messages of the component,
and call the component function with pointers to the message and the callbacks.

The component function reads the call message and performs its specific computation. It prepares the resulting messages of type
\textid{R1CSConstraints}, \textid{AssignedVariables}, or \textid{GadgetReturn}
in memory, and calls the callbacks with pointers to these messages.

A function definition that implements this flow is defined in Listing \ref{gadget.h} as a C header. Refer to the inline documentation for more details.


\parhead{Multi-process execution}

Different parts of the application can be implemented as different
programs, executed separately.

As specified in section~\ref{messagedefinition}, messages are framed and typed,
and can be concatenated in a stream of bytes. It is therefore possible to connect multiple programs through UNIX-style pipes, or to arrange a program to write messages to a file for another program to read later.
To process multiple messages from the same stream or file, a program
reads the 4 bytes containing the size of the next message, allowing it to seek
to the message after that.

A program that constructs a constraint system or implements a gadget should read a GadgetCall message from the standard input stream (stdin).
It should write one or more messages of type
\textid{R1CSConstraints} or \textid{AssignedVariables}
to the standard output stream (stdout).
It should write a single \textid{GadgetReturn} message to the standard error stream
(stderr, used as a control channel).

A program that implements a proving system should read messages of type
\textid{R1CSConstraints}, \textid{AssignedVariables}, or \textid{GadgetReturn}
from stdin, which should contain all the information needed to perform a pre-processing or to generate proofs.


\parhead{Demonstration}

An example implementation is provided.

A number of C++ helper functions are used to interoperate
with libsnark protoboard objects.
A SHA256 gadget from libsnark/gadgetlib1
is encapsulated with the message-based interface of section~\ref{sec:design}.

A test program written in Rust demonstrates
in-memory execution using the method in section~\ref{inmemory},
and includes helper functions to process messages.

This code can be found at \href{https://github.com/QED-it/gadget\_standard}{https://github.com/QED-it/gadget\_standard}.


\lstinputlisting[
	caption=zkinterface.h - C Interface,
	language=C++,style=protobuf,
	label=zkinterface.h]{../cpp/zkinterface.h}


% \begin{thebibliography}{99}
% 	\bibitem{ZKProofSecurity}
% 	1st ZKProof Standards Workshop,
% 	\emph{Security Track Proceeding}.
% 	Online at \href{https://zkproof.org}{ZKProof.org}.
% 	Published in Cambridge, Massachusetts, on May, 2018.
% 
% 	\bibitem{ZKProofImplementation}
% 	1st ZKProof Standards Workshop,
% 	\emph{Implementation Track Proceeding}.
% 	Online at \href{https://zkproof.org}{ZKProof.org}.
% 	Published in Cambridge, Massachusetts, on May, 2018.
% 
% 	\bibitem{ZKProofApplications}
% 	1st ZKProof Standards Workshop,
% 	\emph{Applications Track Proceeding}.
% 	Online at \href{https://zkproof.org}{ZKProof.org}.
% 	Published in Cambridge, Massachusetts, on May, 2018.
% 
% 	\bibitem{zkinterface}
% 	D. Benarroch, K. Gurkan, A. Nicolas, E. Tromer, \emph{gadget\_standard: an implementation of zkInterface}.
% 	Online at \href{https://github.com/QED-it/gadget_standard}{github.com/QED-it/gadget\_standard}.
% 	Tel Aviv, Israel, 2019.
% \end{thebibliography}

\enote{The proceedings bib entries from the bib file are different from the commented-out text above, and also from the FHE-standardization convention.}

\bibliographystyle{alphaurl}
\bibliography{references}
 

%%%%%%%%%%%%%%%%%%%%%%%%%%%%%%%%%%%%%%%%%%%%%%%%%%%%%%%%%%%%%%%%%%%%%%%%%%%%%%%%%%%%%%%%%
\appendix

\section{R1CS Definition}\label{sec:r1cs-def}

\parhead{Background: statement representations}
Many known zk-SNARKs are generic and can express any \emph{NP statement} (i.e., any statement expressed as ``for the instance $x$ there exists a witness $r$ such that relation $D$ holds on $x$ and $w$'' where the relation $D$ is efficiently computable) \cite{ZKProofSecurity}. In principle this already includes convenient high-level representations, and indeed there are research prototypes, for example, for SNARKs supporting general C programs~\cite{BCGTV13csnark-crypto}. However, each zk-SNARK cryptographic construction has a ``native language'' which is a specific type of NP languages. Anything else needs to be converted (reduced) to the native representation, with some overhead. In this set (and upcoming one) we will explore such a native, low-level representation, and show how it can be used to efficiently express useful NP statements to be proven by leading zk-SNARK constructions.

Moreover, expressing NP statements for SNARKs differs from usual programming (whether in JavaScript or assembly) in the following sense. Usual programming aims to \emph{compute} the output of some function of the given inputs, whereas the NP statement's relation is already given the instance (e.g., the claimed output of a function on some input), and aim to \emph{check} its correctness given some additional information from the witness. This affects the programming style and creates opportunities for important optimizations, such as providing additional advice within the witness that aid the checking.

\parhead{Constraint systems}

Today's best zk-SNARK constructions have a native language consisting of \emph{constraint systems}, i.e., a set of constraints on the instance and witness variables.%
%
\footnote{Colloquially, zk-SNARK developers often refer to constraint systems as ``circuits'', but as we shall soon see, the typical constraint system formalism is more general than Boolean or arithmetic circuits.} %
%
The most common such language is \emph{Rank 1 Constraint System (R1CS)}, also known as \emph{Bilinear Constraint Systems (BCS)}, where each constraint is a (bilinear) quadratic equation over a large finite field.  This is the native language of many efficient zk-SNARKs, such as \cite{GGPR13qsp,PHGR13pinocchio,BCGTV13csnark-crypto,BCTV13von,Groth16size} and many others (see \XXX[Security Track document]), and is used in practical deployments.
This style of constraint system was introduced in~\cite{GGPR13qsp}.%
%
\footnote{The initial works~\cite{GGPR13qsp,PHGR13pinocchio} used an equivalent formulation called \emph{Quadratic Arithmetic Programs (QAP)}, where the constraints are expressed in terms of polynomials. Later works tend to abstract this away, and do the conversion into polynomials internally.} %
%
The variant defined below is very similar to the definition in \cite[Appendix~E]{BCGTV13csnark-eprint}, used in \href{https://github.com/scipr-lab/libsnark}{libsnark}.%
%
\footnote{To be precise, ``R1CS'', as defined here, is the same as the ``system of rank-1 quadratic equations'' defined in \cite[Appendix~E]{BCGTV13csnark-eprint}, with a minor change in notation. In defining BCS we let $\vwit$ denote just the witness, whereas \cite{BCGTV13csnark-eprint} let $\mathbf{w}$ denote the concatenation of the instance and the witness (and thus added corresponding consistency checks between $\mathbf{w}$ and the instance $\mathbf{x}$). These are also essentially identical to the ``Rank 1 Constraint System (R1CS)'' of \href{https://github.com/scipr-lab/libsnark}{libsnark}.}

An R1CS reasons about two vectors: the \emph{instance} consisting of $\NumInst$ elements and denoted $\vinst=(\inst_1,\ldots,\inst_\NumInst)$, and the \emph{witness} consisting of $\NumWit$ elements and denoted $\vwit=(\wit_1,\ldots,\wit_\NumWit)$. The constraint system says that the two are related by some number of constraints, $\NumConstr$, each of which is a quadratic equation of a specific form. All of the elements and operations are over a large prime field $\Field$, which we will represent here as the integers modulo a large prime $p$.%
\footnote{The specific prime, and the representation of field elements, are related to the elliptic curve used in the QAP-based zkSNARK constructions.}

\begin{definition}
\label{def:bcs}
Let $\NumInst,\NumWit,\NumConstr$ be positive integers, and $\NumConc=\NumInst+\NumWit+1$.
A \defem{Rank 1 Constraint System (R1CS)}, 
over $\Field$ is a tuple
$\System = \big((\vec{\lcoef}_{j},\vec{\rcoef}_{j},\vec{\ocoef}_{j})_{j=1}^{\NumConstr},\NumInst\big)$
where $\vec{\lcoef}_{j},\vec{\rcoef}_{j},\vec{\ocoef}_{j} \in \Field^{\NumConc}$.
Such a system $\System$ is \defem{satisfiable} with an input $\vinst \in \Field^{\NumInst}$ if the following is true:

\NPstatement
  {For these
    $\System = \big((\vec{\lcoef}_{j},\vec{\rcoef}_{j},\vec{\ocoef}_{j})_{j=1}^{\NumConstr},\NumInst\big)$
   and $\vinst \in \Field^{\NumInst}$}
  {there exists a witness $\vwit \in \Field^{\NumWit}$}
  {such that
  $\bigip{\vec{\lcoef}_{j}}{\vconc} \cdot \bigip{\vec{\rcoef}_{j}}{\vconc}  = \bigip{\vec{\ocoef}_{j}}{\vconc}$ \hskip0.4em for all for $j \in [\NumConstr]$, where $\vconc=(1,\vinst,\vwit)\in \Field^{\NumConc}$
  }
\noindent
Above, $\ip{\cdots}{\cdots}$ denotes inner product of vectors over $\Field$, and $\cdot$ denotes multiplication in $\Field$.
\end{definition}

For example, the R1CS
  \begin{equation}\label{eq:bcs-example-formal}
  \System=\left(\left( 
  \begin{aligned}
   \vec{\lcoef}_1 = (0,3,0,0),\, \vec{\rcoef}_1 = (1,1,0,0),\, \vec{\ocoef}_1 = (4,0,5,0) \\
   \vec{\lcoef}_2 = (0,1,2,0),\, \vec{\rcoef}_2 = (6,0,1,0),\, \vec{\ocoef}_2 = (7,0,0,1)
  \end{aligned}
  \right),1 \right)
  \end{equation}
represents the two bilinear constraints
  \begin{equation}
  \begin{aligned}
  \bigip{(0,3,0,0)}{\vconc}\cdot\bigip{(1,1,0,0)}{\vconc} &= \bigip{(4,0,5,0)}{\vconc} \\
  \bigip{(0,1,2,0)}{\vconc}\cdot\bigip{(6,0,1,0)}{\vconc} &= \bigip{(7,0,0,1)}{\vconc} 
  \end{aligned}
  \quad \mbox{ where } \vconc=(1,\inst_1,\wit_1,\wit_2)
  \end{equation}
or simply:
  \begin{equation}\label{eq:bcs-example-informal}
  \begin{aligned}
  3\inst_1 \cdot (1+\inst_1) &= 5\wit_1 + 4 \\
  (\inst_1 + 2\wit_1) \cdot (\wit_1+6) &= \wit_2 + 7  
  \end{aligned}
  \end{equation}
%with $\NumInst=1,\NumWit=2,\NumConstr=2$

By definition, this system $\System$ is satisfiable with input $\inst_1$ if the following is true:
\NPstatementNOperiod
  {For the instance $\inst_1$}
  {there exist a witness $\wit_1,\wit_2$}
  {such that
  \begin{equation*}
  \begin{aligned}
  3\inst_1 \cdot (1+\inst_1) &= 5\wit_1 + 4 \\
  (\inst_1 + 2\wit_1) \cdot (\wit_1+6) &= \wit_2 + 7  
  \end{aligned}
  \end{equation*}
  }

When we prove this NP statement using zk-SNARK, the instance will be the public input that is visible to the verifer, and the witness will be the additional data that is known only to the prover, and used in the construction of the proof, but whose privacy is protected by the zero-knowledge property.

For backends that consume R1CS natively, typically the cost of producing the zk-SNARK proofs (in time and memory) will depend mostly on the number of constraints, $\NumConstr$. With this in mind, note how variable-by-variable multiplication are ``expensive'' (each bilinear multiplication needs a new constraint and increases $\NumConstr$), but linear combinations come ``for free'' in each constraint. 

\end{document}